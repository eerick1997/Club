\documentclass{resume}

%%%% STYLE %%%%%
\mdfdefinestyle{headerstyle}{%
    linecolor=deepblue,
    outerlinewidth=0.5pt,
    roundcorner=5pt,
    innertopmargin=5pt,
    innerbottommargin=5pt,
    innerrightmargin=5pt,
    innerleftmargin=5pt,
    backgroundcolor=deepblue
}
\setlength{\textwidth}{206mm}
\setlength{\textheight}{300mm}
\setlength{\topmargin}{-36mm}
\setlength{\marginparsep}{1pt}
\begin{document}

\begin{mdframed}[style=headerstyle]
    \name{Erick Efraín VARGAS ROMERO}
    \contact{vargas.erick030997@gmail.com}
    {\link{eerick1997}{https://www.github.com/eerick1997}}
    {\link{eerick1997}{https://www.linkedin.com/in/eerick1997}}
    {(+52) 55 8395 1416}
\end{mdframed}
\vspace*{-\baselineskip}
\resumesection{\faGraduationCap}{Education}
\school{\link{Superior School of Computer Sciences (ESCOM-\link{IPN}{https://www.ipn.mx/}).}{http://www.escom.ipn.mx/}}
{B.S. in Computers System Engineer.}
{Jan. 2016 - expected Jun. 2022}{GPA 87 out of 100 \textbf{(3.3/4.0)}}

\school{\link{Center of Scientific and Technological Studies \#14 (CECyT) `Luis Enrique Erro'.}{https://www.cecyt14.ipn.mx/}}
{Technical track in informatics.}
{2012 - 2015} 
{GPA 81 out of 100 \textbf{(2.7/4.0)}}
\vspace{-0.4cm}
\begin{multicols}{2}        
    \resumesection{\faSearch}{Research}
    \project{Exploration of complex discrete universes using two-dimensional cellular automata (CA).}{
        \begin{itemize}[itemsep=2pt,parsep=2pt]
            \item {Collaborated developing a CA simulator and implementing an algorithm to make the analysis of each rule easier using mean field theory.}
            \item {Collaborated analyzing and exploring random rules of this CA using the simulator previosly mentioned.}
            \item {Participated in the third interpolytechnic meeting of the network of experts in complex systems with the topic ``Complex behavior emerging in two-dimsional discrete dynamic systems''}
        \end{itemize}
    }
    \vspace{0.1cm}
    \resumesection{\faTrophy}{Awards and participation}
    \award{Grand prize of Mexico (ACM-ICPC)}
    {\begin{itemize}[itemsep=2pt,parsep=2pt]
        \item {Participanted with team called \link{`MXerCoders' (2018)}{https://icpc.global/regionals/finder/mcapgp-2018/standings}}
        \item {Participanted with team called `ANSIosos' \link{in place 30 of 316 (2019)}{https://icpc.global/regionals/finder/mcapgp-2019/standings} and \link{in place 41 of 351 (2020)}{https://icpc.global/regionals/finder/mcapgp-2020/standings}}
    \end{itemize}}
    {}
    \vspace*{-\baselineskip}
    \award{\link{Hacking health Monterrey(2018)}{https://hacking-health.org/es/monterrey-es/} }
    {\begin{itemize}[itemsep=2pt,parsep=2pt]
        \item {Participated on Hacking health Monterrey with a project called \link{`Peditriage'}{https://hh-tec-mty-healthathon2018.sparkboard.com/project/5bbb8fa1e4d28a001e63ae75} in a multidisciplinary team up of medical and computer systems engineering students.}
        \item {Developed a mobile application in order to make a pediatric triage more efficient.}
    \end{itemize}}
    {}
\end{multicols}
\vspace{-1.30cm}
\resumesection{\faBriefcase}{Experience}
\experienceitem{Professional Practices}{Banco de México (Bank of Mexico) [Jan 2020 - Aug 2020]}{
    Worked in a system to generate automatically reports getting the information from all the departments using \textbf{SQL Server Reporting Services} (SSRS) that works with \textbf{Visual Basic} and it is powered by \textbf{SQL Server} and \textbf{Sharepoint}.
    \begin{itemize}
        \setlength\itemsep{0cm}
        \item {Software performance was optimized $50\%$ rewriting some recursive code and avoiding consulting a lot of times information.}
        \item {Created new reports that allow analysis of new projects and know the current state of each project.}
    \end{itemize}
}
\resumesection{\faRocket}{Projects}
\project{\link{Virtual Spaces Manager}{https://gitlab.com/ArmandTavera/ingenieriadesoftware3cm2/tree/Arturo-Analisis}}
{This application was developed in order to speed up register virtual spaces of the \link{`Polytechnic Unit for Virtual Education' (UPEV)}{https://www.ipn.mx/upev/} developed using \textbf{JavaScript} and documented using \textbf{\LaTeX}.
\begin{itemize}
    \setlength\itemsep{0cm}
    \item {Collaborated in a team of 30 persons in which I had the role of analyst.}
    \item {Created some documentation to this project including use cases, class diagrams, BPMN diagrams, business rules, etc.}
    \item {Created \textbf{\LaTeX} templates to make the documenting process less cumbersome.}
\end{itemize}
\project{\link{Subgüey}{https://github.com/eerick1997/Subguey}}{
    Mobile application created to make easier to travel by subway or metrobus in Mexico City. Developed using \textbf{Java} for Android to backend and \textbf{XML} to frontend.
    \begin{itemize}
        \setlength\itemsep{0cm}
        \item {Collaborated in a team of 5 persons in which I had the rule of frontend and backend developer.}
        \item {Heavily used and learned Google firebase services.}
    \end{itemize}
}
}

\resumesection{\faCode}{Technical skills}
\vspace*{-\baselineskip}
\begin{multicols}{2}
    \partitle{Programming languages}
    {4 years: Java for Android and Desktop applications.\\
    2 years: C and C++ for Desktop applications.\\
    1 year: C\# for Desktop applications.\\
    Beginner: PHP, JavaScript, Dart, Visual Basic and Python.}
    \break
    \divider
    \partitle{Other technologies}
    \technology{Git, \LaTeX, HTML, CSS, Flutter, JQuery, ReactJS, SQL.}{}
    \break \break \break
    \partitle{Courses}
    Introduction to flutter development by\link{App Brewery.}{https://github.com/eerick1997/Flutter-course/blob/master/certificate.pdf}
    \break
    \divider
    \partitle{Languages}
    \idiom{Spanish (Native) and English (B2 level).}{}
    \divider
    \partitle{Activity}
    \extrainfo{Member of \link{Artificial Life Robotics Lab (ALIROB).}{http://www.comunidad.escom.ipn.mx/ALIROB/Welcome.html}} 
    \extrainfo{Member of \link{ACM student chapter: algorithm club.}{https://www.facebook.com/algoritmiaescom/}}
\end{multicols}
\end{document}