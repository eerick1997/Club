\documentclass{resume}
\begin{document}
\name{Erick Efraín Vargas Romero}
\contact{vargas.erick030997@gmail.com}
{\link{github.com/eerick1997}{https://www.github.com/eerick1997}}
{\link{linkedin.com/in/eerick1997}{https://www.linkedin.com/in/eerick1997}}
{(+52) 5583951416}

\resumesection{\faGraduationCap}{Educación}
\school{\link{Escuela Superior de Cómputo (ESCOM-\link{IPN}{https://www.ipn.mx/})}{http://www.escom.ipn.mx/}}
{Licenciatura en Ingeniería en sistemas computacionales}
{Ene. 2016 - Jul. 2021}{Promedio 8.6 de 10}\break

\school{\link{Centro de estudios científicos y tecnológicos No. 14 (CECyT) Luis Enrique Erro}{https://www.cecyt14.ipn.mx/}}
{Técnico en Informática}
{2012 - 2015}
{Prom. 8.1 de 10}

\resumesection{\faRocket}{Proyectos escolares}
\project{\link{Sübguey}{https://github.com/eerick1997/Subguey}}
{Esta aplicación fue desarollada para personas que viven en ciudad de México y hacen uso de los medios de transporte públicos, sobre todo metro y metrobús los cuales son los más utilizados. Sübguey brinda información en tiempo real sobre el estado de estos medios de transporte, lo cual permite al usuario decidir si es necsario tomar otra ruta alternativa para llegar a su destino. Las tecnologías utilizadas para el desarrollo de esta aplicación fueron Java para Android, XML y algunos servicios de google (Firebase real time database, google auth, google maps). Desarrollada en un equipo de cinco miembros. }

\project{\link{Gestor de espacios virtuales}{https://gitlab.com/ArmandTavera/ingenieriadesoftware3cm2/tree/Arturo-Analisis}}
{Esta aplicación fue desarrollada para eficientar el registro de espacios virtuales de \link{''Unidad Politécnica para la Educación Virtual''}{https://www.ipn.mx/upev/}. Esta aplicación fué desarrollada en un equipo de 28 miembros en el cual tuve el rol de analista.}

\project{\link{Digital Portraits}{https://github.com/eerick1997/ProjectCryptography}}
{Desarrollada para enviar y recibir imágenes pero utilizando algunos servicios cryptográficos como son: autenticación, no repudio e integridad de datos. Esta aplicación fue desarrollada utilizando Java para Android y algunos servicios de Google. Desarrollada en equipo de cuatro miembros. }

\resumesection{\faTrophy}{Premios}
\award{\link{Haking health monterrey}{https://hacking-health.org/es/monterrey-es/} }
{Participante con el proyecto llamado ''Peditriage''}
{2018}
\award{\link{Gran premio de México (ACM-ICPC)}{https://icpc.baylor.edu/regionals/finder/mcapgp-2018/standings}}
{Participante con el equipo llamado ''MXerCoders'' (2018) y posteriormente con el equipo ''ANSIosos' finalizando en lugar 30 de 316 (2019)}
{}
\resumesection{\faCode}{Habilidades técnicas}
\begin{multicols}{2}
    \partitle{Lenguajes de programación}
    \programminglanguage{Java}{4 año}
    \programminglanguage{C / C++}{3 año}
    \programminglanguage{C\# VHDL}{1 año}
    \programminglanguage{PHP, JavaScript, python}{3 meses}
    \divider
    \partitle{Otras tecnologías}
    \technology{Git, LaTex, HTML, CSS, Java andoid, nodejs(básico), reactjs(básico)}{}
    \technology{SQL, Fire base realtime database}{}
    \partitle{Plataformas}
    \platform{Windows-10(usuario)}
    \platform{Linux (Manjaro usuario)} \break
    \divider
    \partitle{Idiomas}
    \idiom{Español}{Nativo}
    \idiom{Inglés}{Nivel B2}
\end{multicols}
\resumesection{\faStar}{Áreas de interes}
Análisis de software, criptografia, backend.
\end{document}