We already know the behaviour of a linked list, and if we are progrogramming a Singly linked list it means that we need just information about the next node and the value of the current node, so we need to use a struct called Node with a pointer to the next element, and a int value (the value of the current node), if we code this (and knowing the basic operations, explained in the section Basic Structures) we have the next as a result:

\begin{figure}[H]
    \centering
    \includegraphics[width=1.00\textwidth]{Images/DataStructures/LinkedLists/SinglyLinkedList.png}
    \caption{Diagram of a singly linked list}
    \label{fig:singly_linked_list_diagram-01}
\end{figure}

\begin{lstlisting}
    struct Node{

        Node *next;
        int value;

        Node( int _value ){
            value = _value;
            next = NULL;
        }

    };

    typedef struct LinkedList{
        
        Node *tail;
        Node *head;
        int size;

        LinkedList(){
            tail = head = NULL;
            size = 0;
        }

        void insertAtTail(int value){
            Node *node = new Node( value );
            if( size > 0 ){
                tail -> next = node;
                tail = node;
            } else {
                tail = head = node;
            }
            size++;
        }

        void insertAtHead(int value){
            Node *node = new Node( value );
            if( size > 0 ){ 
                node -> next = head;
                head = node;
            } else {
                head = tail = node;
            }
            size++;
        }

        void Delete(int i){
            if( i < size && i >= 0 ){
                Node *aux = head;
                //If we want to delete the first element
                if( i == 0 )
                    head = head -> next;
                //If we want to delete the last element
                else if ( i == size ){

                    for( int j = 0; j <= size - 1; j++ )
                        aux = aux -> next;

                    aux -> next = NULL;
                    tail = aux;
        
                } else {
                    for( int j = 0; j < i - 1; j++)
                        aux = aux -> next;
                    aux -> next = aux -> next -> next;
                } 
                size--;
            }
        }

        void DeleteAtHead(){
            if( size == 1 )
                head = tail = head -> next; 
            else
                head = head -> next;
            size--;
        }

        int Search(int i){

            if( i < size ){
                Node *aux = head;
                for(int j = 0; j < i; j++)
                    aux = aux -> next;
                return aux -> value;
            }

            return INT_MIN;
        }

        bool isEmpty(){
            return ( size == 0 );
        }

        int getSize(){
            return size;
        }

        void print(){
            Node *aux = head;
            while( aux != NULL ){
                cout << aux -> value << "\t";
                aux = aux -> next;
            }
            cout << endl;
        }
    } LinkedList;
\end{lstlisting}

\subsection{Problems}
\subsubsection{Problem 01}
\textsf{Reverse a linked list}
To solve this problem we need to know what is a stack, and I already explain this data structure, so we only need to remember this behaviour to solve this problem.\\
First of all I have created a struct with all the basic operations of a Stack, but instead of nodes this struct contains a Queue.
Our operation of pop does not have any problem but oir operation of push needs to be changed, so invert the elements inside our Queue is a goos solution for this problem, and we can do this using two queues.
To solve the problem of push an element as I mentioned before we need two queues, the first one contains the elements of our stack, and the second we go to use it when we need to add a new element, the procedure is the following:
First add the new element to our second queue then make a dequeue of our first queue until this queue be empty, finally we made the second queue as our first queue. In code we have the next as a result:

\begin{lstlisting}
    #include <bits/stdc++.h>

    using namespace std;

    struct Stack{

        int size;
        queue<int> Queue;

        Stack(){
            size = 0;
        }

        void Push( int val ){
            queue<int> result;
            result.push( val );
            while( !Queue.empty() ){
                result.push( Queue.front() );
                Queue.pop();
            }
            Queue = result;
            size++;
        }

        void Pop(){
            if( size > 0 ){
                Queue.pop();
                size--;
            } else {
                cout << "\nStack doesn't have elements\n";
            }
        }

        int Top(){
            return Queue.front();
        }

        int GetSize(){
            return size;
        }

        bool IsEmpty(){
            return ( size == 0 );
        }
    };
\end{lstlisting}


\subsubsection{Problem 02}
\textsf{Detect a loop in a linked list}
First of all you need to know that if you are in an interview you need to ask if you can use another data structure, and other constraints, in this case imagine that you can use another data structure apart of queues, so if we need to reverse elements we have a data structure that can help us to do this easily and it is a stack.
We need to pop the first k elements of our queue and store this elements in a stack, then make a pop of our stack and store the elements in our queue, and this elements now are reversed because the behaviour of a stack help us to do this. Finally we need to ''nove'' n - k times the elements inside our queue and thats all, in code we have the next as a result:
\begin{lstlisting}
    #include <bits/stdc++.h>

    using namespace std;

    int main(){
        queue<int> Queue;
        stack<int> Stack;
        int n, v, k;
        cin >> n;
        while(n--){
            cin >> v;
            Queue.push(v);
        }

        cin >> k;

        for (int i = 0; i < k; i++){
            Stack.push( Queue.front() );
            Queue.pop();
        }

        while( !Stack.empty() ){
            Queue.push( Stack.top() );
            Stack.pop();
        }

        for(int i = 0; i < Queue.size() - k; i++){
            Queue.push( Queue.front() );
            Queue.pop();
        }

        while( !Queue.empty() ){
            cout << Queue.front() << " ";
            Queue.pop();
        }
        
        return 0;
    }
\end{lstlisting}

\subsubsection{Problem 03}
\textsf{Return N-th node from the end in a linked list}
Solve this problem is not difficult just you need to check which element between our two arrays is smaller than the other, and we will do this until one of our indexes is in the limit. Finally we need to check if one of our arrays was not checked completely. 
\begin{lstlisting}
    #include <bits/stdc++.h>

    using namespace std;

    int main(){

        int m, n;
        vector<int> M, N, ans;

        cin >> m >> n;
        M.resize(m, 0);
        N.resize(n, 0);

        for(int i = 0; i < m; i++)
            cin >> M[i];

        for(int i = 0; i < n; i++)
            cin >> N[i];

        int i = 0, j = 0;
        while( i < m && j < n ){
            if(M[i] < N[j]){
                ans.push_back( M[i++] );
            } else {
                ans.push_back( N[j++] );
            }
        }

        for ( ; i < m; i++)
            ans.push_back( M[i] );

        for( ; j < n; j++)
            ans.push_back( N[j] );

        for( auto e : ans )
            cout << e << " ";
        
        cout << endl;
        return 0;
    }
\end{lstlisting}

\subsubsection{Problem 04}
\textsf{Remove duplicates from a linked list}
To solve this problem we need something to help us, maybe we can use a bucket, but maybe we do not know the biggest number of an element inside the linked list, and also it makes to use a lot of memory so we can use hash to make this efficient in time and memory, in this case I gonna use an unordered set, it is a contianter that uses a hash to store efficientlly data, if you need more information about this you can show the \href{https://en.cppreference.com/w/cpp/container/unordered_set}{ \textbf{CPP reference} } and here is the solution for this problem.
\begin{lstlisting}
    #include <bits/stdc++.h>

    using namespace std;

    struct Node{
        Node *next;
        int value;
        Node(){
            next = NULL;
        }
    };

    Node *RemoveDuplicates( Node *head ){
        unordered_set<int> duplicates;
        Node *current = head;
        Node *previous = NULL;
        while(current -> next != NULL){

            if( duplicates.find( current -> value ) != duplicates.end() ){
                previous -> next = current -> next;
            } else {
                duplicates.insert( current -> value );
                previous = current;
            }
            current = current -> next;
        }

        return head;
    }

    void printList( Node *head ){
        Node *aux = head;
        cout << endl;
        while( aux -> next != NULL ){
            cout << aux -> value << " ";
            aux = aux -> next;
        }
        cout << endl;
    }

    int main(){
        int n, v;
        Node *head = new Node();
        Node *aux = head;

        cin >> n; 
        while(n--){
            cin >> aux -> value;
            aux -> next = new Node();
            aux = aux -> next;
        }

        printList( head );

        aux = RemoveDuplicates( head );

        printList( aux );
        
        return 0;
    }
\end{lstlisting}
