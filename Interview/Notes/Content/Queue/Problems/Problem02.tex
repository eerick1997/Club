First of all you need to know that if you are in an interview you need to ask if you can use another data structure, and other constraints, in this case imagine that you can use another data structure apart of queues, so if we need to reverse elements we have a data structure that can help us to do this easily and it is a stack.
We need to pop the first k elements of our queue and store this elements in a stack, then make a pop of our stack and store the elements in our queue, and this elements now are reversed because the behaviour of a stack help us to do this. Finally we need to ''nove'' n - k times the elements inside our queue and thats all, in code we have the next as a result:
\begin{lstlisting}
    #include <bits/stdc++.h>

    using namespace std;

    int main(){
        queue<int> Queue;
        stack<int> Stack;
        int n, v, k;
        cin >> n;
        while(n--){
            cin >> v;
            Queue.push(v);
        }

        cin >> k;

        for (int i = 0; i < k; i++){
            Stack.push( Queue.front() );
            Queue.pop();
        }

        while( !Stack.empty() ){
            Queue.push( Stack.top() );
            Stack.pop();
        }

        for(int i = 0; i < Queue.size() - k; i++){
            Queue.push( Queue.front() );
            Queue.pop();
        }

        while( !Queue.empty() ){
            cout << Queue.front() << " ";
            Queue.pop();
        }
        
        return 0;
    }
\end{lstlisting}