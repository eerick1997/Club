As the Stack we already know the basic operations of a Queue, so we have an idea about what we go to program, but what is the structure of a Queue? As I mentioned before maybe show a draw or schema is better to understand what we need to program so I have maked a little draw of how I imagine a Queue.

\begin{figure}[H]
    \centering
    \includegraphics[width=1.00\textwidth]{Images/DataStructures/Queue/Queue.png}
    \caption{Diagram of a Queue}
    \label{fig:queue_diagram-01}
\end{figure}

As you can see a stack has a ''back'' and ''front'' element where ''back'' is the last element of the queue and ''front'' is the first element. Each element (lets call them as nodes) has a value and a pointer to the next element, if we wrote this using code we have the next result:

\begin{lstlisting}
    #include <bits/stdc++.h>

    using namespace std;

    struct Node{
        int value;
        Node *next;

        Node( int _value ){
            value = _value;
            next = NULL;
        }
    };

    struct Queue{

        Node *back;
        Node *front;
        int size;

        Queue(){
            back = front = NULL;
            size = 0;
        }

        void Enqueue( int value ){
            Node *node = new Node( value );
            if( size == 0 )
                back = front = node;
            else {
                node -> next = back;
                back = node;
            }

            size++;
        }

        void Dequeue(){
            Node *aux = back;
            if( size > 1 ){
                
                while( aux -> next != NULL )
                    aux = aux -> next;

                aux -> next = NULL;
                front = aux;

            } else if( size == 1 )
                front = back = NULL;
            else 
                return;
            size--;
        }

        bool IsEmpty(){
            return (size == 0);
        }

        int Top(){
            return (front -> value);
        }

        int GetSize(){
            return size;
        }
        
        void print(){
            cout << endl;
            Node *aux = back;
            while( aux != NULL ){
                cout << aux -> value << " ";
                aux = aux -> next;
            }
            cout << endl;
        }
    };
\end{lstlisting}

\subsection{Problems}
\subsubsection{Problem 01}
\textsf{Implement a stack using a queue}
To solve this problem we need to know what is a stack, and I already explain this data structure, so we only need to remember this behaviour to solve this problem.\\
First of all I have created a struct with all the basic operations of a Stack, but instead of nodes this struct contains a Queue.
Our operation of pop does not have any problem but oir operation of push needs to be changed, so invert the elements inside our Queue is a goos solution for this problem, and we can do this using two queues.
To solve the problem of push an element as I mentioned before we need two queues, the first one contains the elements of our stack, and the second we go to use it when we need to add a new element, the procedure is the following:
First add the new element to our second queue then make a dequeue of our first queue until this queue be empty, finally we made the second queue as our first queue. In code we have the next as a result:

\begin{lstlisting}
    #include <bits/stdc++.h>

    using namespace std;

    struct Stack{

        int size;
        queue<int> Queue;

        Stack(){
            size = 0;
        }

        void Push( int val ){
            queue<int> result;
            result.push( val );
            while( !Queue.empty() ){
                result.push( Queue.front() );
                Queue.pop();
            }
            Queue = result;
            size++;
        }

        void Pop(){
            if( size > 0 ){
                Queue.pop();
                size--;
            } else {
                cout << "\nStack doesn't have elements\n";
            }
        }

        int Top(){
            return Queue.front();
        }

        int GetSize(){
            return size;
        }

        bool IsEmpty(){
            return ( size == 0 );
        }
    };
\end{lstlisting}


\subsubsection{Problem 02}
\textsf{Reverse first k elements of a queue}
First of all you need to know that if you are in an interview you need to ask if you can use another data structure, and other constraints, in this case imagine that you can use another data structure apart of queues, so if we need to reverse elements we have a data structure that can help us to do this easily and it is a stack.
We need to pop the first k elements of our queue and store this elements in a stack, then make a pop of our stack and store the elements in our queue, and this elements now are reversed because the behaviour of a stack help us to do this. Finally we need to ''nove'' n - k times the elements inside our queue and thats all, in code we have the next as a result:
\begin{lstlisting}
    #include <bits/stdc++.h>

    using namespace std;

    int main(){
        queue<int> Queue;
        stack<int> Stack;
        int n, v, k;
        cin >> n;
        while(n--){
            cin >> v;
            Queue.push(v);
        }

        cin >> k;

        for (int i = 0; i < k; i++){
            Stack.push( Queue.front() );
            Queue.pop();
        }

        while( !Stack.empty() ){
            Queue.push( Stack.top() );
            Stack.pop();
        }

        for(int i = 0; i < Queue.size() - k; i++){
            Queue.push( Queue.front() );
            Queue.pop();
        }

        while( !Queue.empty() ){
            cout << Queue.front() << " ";
            Queue.pop();
        }
        
        return 0;
    }
\end{lstlisting}

\subsubsection{Problem 03}
\textsf{Generate binary numbers from 1 to n using a queue}
Solve this problem is not difficult just you need to check which element between our two arrays is smaller than the other, and we will do this until one of our indexes is in the limit. Finally we need to check if one of our arrays was not checked completely. 
\begin{lstlisting}
    #include <bits/stdc++.h>

    using namespace std;

    int main(){

        int m, n;
        vector<int> M, N, ans;

        cin >> m >> n;
        M.resize(m, 0);
        N.resize(n, 0);

        for(int i = 0; i < m; i++)
            cin >> M[i];

        for(int i = 0; i < n; i++)
            cin >> N[i];

        int i = 0, j = 0;
        while( i < m && j < n ){
            if(M[i] < N[j]){
                ans.push_back( M[i++] );
            } else {
                ans.push_back( N[j++] );
            }
        }

        for ( ; i < m; i++)
            ans.push_back( M[i] );

        for( ; j < n; j++)
            ans.push_back( N[j] );

        for( auto e : ans )
            cout << e << " ";
        
        cout << endl;
        return 0;
    }
\end{lstlisting}