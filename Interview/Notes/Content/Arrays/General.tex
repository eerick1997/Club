We already known what is an array, and now we need to know how to solve some problmes using this data structure. First of all we need to know how to use arrays in a programming language, in my case I used to use C++ language because is very easy to understand and is faster than a lot of programming languages.

In C++ we can declarate an array using the next piece of code.
\begin{lstlisting}
    data_type name_of_our_array[ size_of_our_array ]
\end{lstlisting}

Imagine that you need an array of integer numbers with length ten, we need to write the next in out C++ programm.

\begin{lstlisting}
    #include<bits/stdc++.h>
    
    using namespace std;

    int main(){
        //As you can see this is the form of create an array of integers with length ten
        int array[10];
        return 0;
    }
\end{lstlisting}

If we need to set a value in a element of our array we just need to put the index of our element, but consider that arrays indexes starts with 0

\begin{lstlisting}
    #include <bits/stdc++.h>
    
    using namespace std;

    int main(){
        
        int array[10];
        array[0] = 1;
        array[1] = 2;
        array[3] = 3;
        .
        .
        .
        array[9] = 10;
        return 0;
    }

\end{lstlisting}

If we want to get the value of an element of our array we need just to indicate the index of our element, something like this:
\begin{lstlisting}
    #include <bits/stdc++.h>

    using namespace std;

    int main(){
        int array[10] = {1, 2, 3, 4, 5, 6, 7, 8, 9, 10};
        cout << array[2] << endl;
        return 0;
    }

    Output: 3
\end{lstlisting}

\subsection{Some problems}
\subsubsection{Problem 01}
\textsf{Find the second minimum element of an array}
To solve this problem we need to know what is a stack, and I already explain this data structure, so we only need to remember this behaviour to solve this problem.\\
First of all I have created a struct with all the basic operations of a Stack, but instead of nodes this struct contains a Queue.
Our operation of pop does not have any problem but oir operation of push needs to be changed, so invert the elements inside our Queue is a goos solution for this problem, and we can do this using two queues.
To solve the problem of push an element as I mentioned before we need two queues, the first one contains the elements of our stack, and the second we go to use it when we need to add a new element, the procedure is the following:
First add the new element to our second queue then make a dequeue of our first queue until this queue be empty, finally we made the second queue as our first queue. In code we have the next as a result:

\begin{lstlisting}
    #include <bits/stdc++.h>

    using namespace std;

    struct Stack{

        int size;
        queue<int> Queue;

        Stack(){
            size = 0;
        }

        void Push( int val ){
            queue<int> result;
            result.push( val );
            while( !Queue.empty() ){
                result.push( Queue.front() );
                Queue.pop();
            }
            Queue = result;
            size++;
        }

        void Pop(){
            if( size > 0 ){
                Queue.pop();
                size--;
            } else {
                cout << "\nStack doesn't have elements\n";
            }
        }

        int Top(){
            return Queue.front();
        }

        int GetSize(){
            return size;
        }

        bool IsEmpty(){
            return ( size == 0 );
        }
    };
\end{lstlisting}


\subsubsection{Problem 02}
\textsf{First non-repeating integers in the array}
First of all you need to know that if you are in an interview you need to ask if you can use another data structure, and other constraints, in this case imagine that you can use another data structure apart of queues, so if we need to reverse elements we have a data structure that can help us to do this easily and it is a stack.
We need to pop the first k elements of our queue and store this elements in a stack, then make a pop of our stack and store the elements in our queue, and this elements now are reversed because the behaviour of a stack help us to do this. Finally we need to ''nove'' n - k times the elements inside our queue and thats all, in code we have the next as a result:
\begin{lstlisting}
    #include <bits/stdc++.h>

    using namespace std;

    int main(){
        queue<int> Queue;
        stack<int> Stack;
        int n, v, k;
        cin >> n;
        while(n--){
            cin >> v;
            Queue.push(v);
        }

        cin >> k;

        for (int i = 0; i < k; i++){
            Stack.push( Queue.front() );
            Queue.pop();
        }

        while( !Stack.empty() ){
            Queue.push( Stack.top() );
            Stack.pop();
        }

        for(int i = 0; i < Queue.size() - k; i++){
            Queue.push( Queue.front() );
            Queue.pop();
        }

        while( !Queue.empty() ){
            cout << Queue.front() << " ";
            Queue.pop();
        }
        
        return 0;
    }
\end{lstlisting}

\subsubsection{Problem 03}
\textsf{Merge two sorted arrays}
Solve this problem is not difficult just you need to check which element between our two arrays is smaller than the other, and we will do this until one of our indexes is in the limit. Finally we need to check if one of our arrays was not checked completely. 
\begin{lstlisting}
    #include <bits/stdc++.h>

    using namespace std;

    int main(){

        int m, n;
        vector<int> M, N, ans;

        cin >> m >> n;
        M.resize(m, 0);
        N.resize(n, 0);

        for(int i = 0; i < m; i++)
            cin >> M[i];

        for(int i = 0; i < n; i++)
            cin >> N[i];

        int i = 0, j = 0;
        while( i < m && j < n ){
            if(M[i] < N[j]){
                ans.push_back( M[i++] );
            } else {
                ans.push_back( N[j++] );
            }
        }

        for ( ; i < m; i++)
            ans.push_back( M[i] );

        for( ; j < n; j++)
            ans.push_back( N[j] );

        for( auto e : ans )
            cout << e << " ";
        
        cout << endl;
        return 0;
    }
\end{lstlisting}

\subsubsection{Problem 04}
\textsf{Rearrage positive and negative values in an array}